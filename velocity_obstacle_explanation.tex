\documentclass[12pt,a4paper]{article}
\usepackage[utf8]{inputenc}
\usepackage[brazil]{babel}
\usepackage{amsmath}
\usepackage{amsfonts}
\usepackage{amssymb}
\usepackage{graphicx}
\usepackage{listings}
\usepackage{xcolor}
\usepackage{hyperref}
\usepackage{booktabs}

\title{Análise Acadêmica do Algoritmo Velocity Obstacle\\
\large Implementação em ColAvd\_vo.cpp}
\author{Baseado em Fiorini \& Shiller (1998)}
\date{}

\lstset{
    language=C++,
    basicstyle=\ttfamily\small,
    keywordstyle=\color{blue},
    commentstyle=\color{green!60!black},
    numbers=left,
    numberstyle=\tiny,
    frame=single,
    breaklines=true
}

\begin{document}

\maketitle

\section{Fundamento Teórico}

A implementação em \texttt{ColAvd\_vo.cpp} baseia-se no conceito de \textbf{Velocity Obstacle (VO)} proposto por Fiorini e Shiller (1998), que representa o conjunto de velocidades do veículo próprio que resultarão em colisão com um obstáculo móvel.

\subsection{Definição do Velocity Obstacle}

Matematicamente, o VO é definido como:

\begin{equation}
VO(A,B) = \left\{ \mathbf{v}_A \mid \exists t \in [0,\infty) : \|\mathbf{p}_A + t\mathbf{v}_A - (\mathbf{p}_B + t\mathbf{v}_B)\| \leq r_A + r_B \right\}
\end{equation}

Onde:
\begin{itemize}
    \item $\mathbf{v}_A$: velocidade do veículo próprio
    \item $\mathbf{p}_A, \mathbf{p}_B$: posições dos veículos $A$ e $B$
    \item $\mathbf{v}_B$: velocidade do obstáculo
    \item $r_A, r_B$: raios de segurança
\end{itemize}

\section{Implementação do Collision Cone}

\subsection{Cálculo do Cone de Colisão}

A função \texttt{computeCollisionCone()} (linhas 345--373) calcula o \textbf{cone de colisão} no espaço de configuração:

\begin{lstlisting}
// Angulo ate o contato
double theta_to_contact = 90.0 - atan2(dy, dx) * 180.0 / M_PI;

// Semi-angulo do cone (linha 359)
double alpha = asin(m_safety_radius / dist) * 180.0 / M_PI;

// Limites do cone
double cone_left = theta_to_contact - alpha;
double cone_right = theta_to_contact + alpha;
\end{lstlisting}

Este cálculo implementa a \textbf{geometria fundamental do VO}: o semi-ângulo $\alpha$ é derivado da relação geométrica entre o raio de segurança $r$ e a distância ao obstáculo $d$:

\begin{equation}
\alpha = \arcsin\left(\frac{r_{\text{safety}}}{d}\right)
\end{equation}

\subsection{Translação para o Espaço de Velocidades}

O algoritmo realiza a translação do cone de colisão para o espaço de velocidades (linhas 460--481):

\begin{equation}
VO^\tau_{A|B} = \mathbf{v}_B \oplus CC_{A|B}
\end{equation}

Onde:
\begin{itemize}
    \item $\oplus$: operação de soma de Minkowski
    \item $CC_{A|B}$: collision cone centrado na origem
    \item $\mathbf{v}_B$: velocidade do obstáculo
\end{itemize}

Implementação:
\begin{lstlisting}
// Apice do VO transladado pela velocidade do contato
double vo_apex_x = vo_origin_x + cv_x * vel_scale;
double vo_apex_y = vo_origin_y + cv_y * vel_scale;
\end{lstlisting}

\section{Integração com COLREGS}

\subsection{Classificação de Situações}

O algoritmo implementa três cenários das regras COLREGS (linhas 277--308):

\subsubsection{Head-on (linhas 280--286)}
\begin{itemize}
    \item Ambos os veículos em setores de proa mútuos
    \item Diferença de heading $\approx 180^\circ$ ($\pm 10^\circ$)
\end{itemize}

\subsubsection{Overtaking (linhas 291--298)}
\begin{itemize}
    \item Veículo próprio no setor de popa do contato ($\pm 67.5^\circ$ da popa)
    \item Contato no setor de proa do próprio
\end{itemize}

\subsubsection{Crossing (linhas 302--308)}
\begin{itemize}
    \item Contato no setor de proa do próprio
    \item Não é head-on nem overtaking
\end{itemize}

\subsection{Lógica de Ativação}

O sistema usa \textbf{activation logic baseada em setores} conforme Cho et al. (2019):

\begin{lstlisting}
bool contact_in_own_bow = (fabs(relative_bearing_own) < 90.0);
bool own_in_contact_bow = (fabs(relative_bearing_contact) < 90.0);
\end{lstlisting}

Os marcadores booleanos determinam a posição relativa entre os veículos:
\begin{equation}
\text{bow\_sector} = \begin{cases}
\text{true}, & \text{se } |\beta| < 90^\circ \\
\text{false}, & \text{caso contrário}
\end{cases}
\end{equation}

onde $\beta$ é o bearing relativo.

\section{Seleção de Velocidade Livre de Colisão}

\subsection{Estratégia de Desvio}

A implementação atual utiliza uma \textbf{estratégia simplificada de desvio por boreste} (direita), conforme linhas 374--380:

\begin{lstlisting}
best_heading = cone_right + 2.0; // Margem de seguranca de 2 graus
\end{lstlisting}

Matematicamente:
\begin{equation}
\theta_{\text{best}} = \theta_{\text{contact}} + \alpha + \Delta\theta_{\text{margin}}
\end{equation}

onde $\Delta\theta_{\text{margin}} = 2^\circ$.

Esta abordagem é consistente com a Regra 14 das COLREGS (situação head-on).

\subsection{Limitações da Implementação Atual}

Diferentemente do algoritmo completo de Fiorini \& Shiller, que realiza:
\begin{enumerate}
    \item Amostragem do espaço de velocidades
    \item Otimização por função de custo
    \item Seleção da velocidade admissível mais próxima da desejada
\end{enumerate}

A implementação atual (linhas 320--324) não executa a \textbf{busca completa no espaço de velocidades}. O algoritmo original propõe:

\begin{equation}
\mathbf{v}^* = \arg\min_{\mathbf{v} \notin VO} \|\mathbf{v} - \mathbf{v}_{\text{des}}\|
\end{equation}

onde $\mathbf{v}_{\text{des}}$ é a velocidade desejada.

\section{Visualização do Espaço de Velocidades}

\subsection{Representação Dual}

O código implementa visualização em dois espaços (linhas 398--511):

\subsubsection{Espaço de Configuração (linhas 432--449)}
\begin{itemize}
    \item Cone vermelho emanando do veículo próprio
    \item Representa regiões proibidas de movimento no espaço $(x, y)$
\end{itemize}

\subsubsection{Espaço de Velocidades (linhas 460--481)}
\begin{itemize}
    \item Cone laranja transladado pela velocidade do obstáculo
    \item Origem deslocada em $(200, 200)$ para visualização
    \item Fator de escala: $\lambda = 20$ (linha 402)
\end{itemize}

Transformação de coordenadas:
\begin{align}
x_{\text{vis}} &= x_{\text{origin}} + v_x \cdot \lambda \\
y_{\text{vis}} &= y_{\text{origin}} + v_y \cdot \lambda
\end{align}

\subsubsection{Velocidade Selecionada (linhas 501--511)}
\begin{itemize}
    \item Vetor verde indicando a velocidade comandada
    \item Comprimento proporcional à magnitude da velocidade
\end{itemize}

\section{Comparação com Fiorini \& Shiller}

\begin{table}[h]
\centering
\begin{tabular}{@{}lcc@{}}
\toprule
\textbf{Aspecto} & \textbf{Fiorini \& Shiller} & \textbf{Implementação Atual} \\ \midrule
Cálculo do VO & \checkmark Completo & \checkmark Cone básico \\
Translação $\mathbf{v}_B$ & \checkmark Explícita & \checkmark Visualização apenas \\
Busca no espaço de $\mathbf{v}$ & \checkmark Amostragem + otim. & $\times$ Desvio fixo \\
Múltiplos obstáculos & \checkmark União de VOs & $\sim$ Trata o mais próximo \\
Função de custo & \checkmark Minimização & $\times$ Não implementada \\ \bottomrule
\end{tabular}
\caption{Comparação entre a teoria original e a implementação}
\end{table}

\section{Análise de Complexidade}

\subsection{Complexidade Temporal}

Para $n$ contatos:
\begin{itemize}
    \item \textbf{Classificação COLREGS}: $O(n)$ --- linhas 238--310
    \item \textbf{Identificação do mais próximo}: $O(n)$ --- linhas 330--343
    \item \textbf{Cálculo do cone}: $O(1)$ --- linhas 345--380
    \item \textbf{Visualização}: $O(n)$ --- linhas 398--511
\end{itemize}

Complexidade total: $O(n)$

\subsection{Complexidade Espacial}

\begin{itemize}
    \item Armazenamento de contatos: $O(n)$ no mapa \texttt{m\_contacts}
    \item Estruturas de visualização: $O(n)$ para segmentos de linha
\end{itemize}

\section{Conclusão}

A implementação representa uma \textbf{versão simplificada e híbrida} do VO clássico:

\begin{itemize}
    \item \textbf{Fundamento teórico}: Mantém a geometria do cone de colisão de Fiorini \& Shiller (Eq. 2)
    \item \textbf{Estratégia de desvio}: Utiliza regra heurística COLREGS ao invés de otimização no espaço de velocidades
    \item \textbf{Decisão de manobra}: Baseada em setores náuticos ao invés de análise puramente geométrica do VO
\end{itemize}

Esta abordagem prioriza \textbf{conformidade com regulações marítimas} sobre a solução matematicamente ótima do VO original, sendo adequada para aplicações em veículos de superfície não tripulados (USVs) que operam em ambientes regulamentados.

\subsection{Trabalhos Futuros}

Para uma implementação completa do algoritmo de Fiorini \& Shiller, sugere-se:

\begin{enumerate}
    \item Implementar amostragem sistemática do espaço de velocidades
    \item Adicionar função de custo multi-objetivo:
    \begin{equation}
    J(\mathbf{v}) = \alpha_1 \|\mathbf{v} - \mathbf{v}_{\text{des}}\| + \alpha_2 \cdot d_{\text{clearance}}(\mathbf{v}) + \alpha_3 \cdot \text{COLREGS\_penalty}(\mathbf{v})
    \end{equation}
    \item Tratar múltiplos obstáculos através da união de VOs:
    \begin{equation}
    VO_{\text{total}} = \bigcup_{i=1}^{n} VO_{A|B_i}
    \end{equation}
    \item Implementar predição de trajetórias para horizontes temporais finitos
\end{enumerate}

\begin{thebibliography}{9}

\bibitem{fiorini1998}
P. Fiorini and Z. Shiller,
``Motion Planning in Dynamic Environments Using Velocity Obstacles,''
\textit{The International Journal of Robotics Research}, vol. 17, no. 7, pp. 760--772, 1998.

\bibitem{cho2019}
Y. Cho et al.,
``Experimental validation of a velocity obstacle based collision avoidance algorithm for unmanned surface vehicles,''
\textit{IFAC-PapersOnLine}, vol. 52, no. 21, pp. 329--334, 2019.

\bibitem{colregs}
International Maritime Organization,
``Convention on the International Regulations for Preventing Collisions at Sea (COLREGS),'' 1972.

\end{thebibliography}

\end{document}
